% !Mode:: "TeX:UTF-8"
\documentclass[QAofGroup.tex]{subfiles}

\begin{document}
%-=-=-=-=-=-=-=-=-=-=-=-=-=-=-=-=-=-=-=-=-=-=-=-=
%
%	CHAPTER
%
%-=-=-=-=-=-=-=-=-=-=-=-=-=-=-=-=-=-=-=-=-=-=-=-=

%%================================================================
\chapter{20180414}\label{ch180414}
%----------------------------------------------------------------------------------------
\begin{qst}\label{Q2018041401}
	如何绘制流程图\index{流程图}
\end{qst}
\ans 使用TiKZ和Pstricks
\begin{minted}[breaklines]{tex}
\documentclass{ctexart}
\usepackage{tikz}
\usetikzlibrary{shapes.geometric}
\usepackage{setspace}

% 流程图定义基本形状
\tikzstyle{startstop} = [rectangle, rounded corners, minimum width = 1em, minimum height=0.8cm,text centered, draw = black]
\tikzstyle{io} = [trapezium, trapezium left angle=80, trapezium right angle=100, minimum width=1.5cm, minimum height=0.8cm, text centered, draw=black]
\tikzstyle{process} = [rectangle, minimum width=1em, minimum height=0.8cm, text centered, draw=black]
\tikzstyle{decision} = [diamond, aspect = 0.3, text centered, draw=black]
% 箭头形式
\tikzstyle{arrow} = [->,>=stealth]

\begin{document}

\begin{figure}[!htb]
\centering
\zihao{5}


\begin{tikzpicture}[node distance=0.5cm]
%定义流程图具体形状
\node[startstop](start){\parbox{1em}{初始结构}};
\node[process, right of = start, xshift = 1.6cm](ganxi){\parbox{1em}{杆系结构\end{spacing}}};
\node[process, right  of = ganxi, xshift = 1.8cm](lianxuti){\parbox{1em}{连续体结构}};
\node[decision, right of = lianxuti, xshift = 4.5cm](sltj){\parbox{1em}{控制条件}};
\node[startstop, right of = sltj, xshift = 2.5cm](out1){\parbox{1em}{新结构}};
\coordinate (point1) at (6cm, -3cm);
\coordinate (point2) at (3.15cm, 0cm);

%连接具体形状
\draw [arrow] (start) -- (ganxi);
\draw [arrow] (ganxi) -- (lianxuti);
\draw [arrow] (lianxuti) --node [above] {消除、增殖} (sltj);
\draw [arrow](sltj) --node [above] {是} (out1);
\draw (sltj) |-node [above,xshift=-3.5cm] {否} (point1);
\draw [arrow](point1) -| (point2);
\end{tikzpicture}
\vspace{0.5em}
\caption{基本思路流程图}
\label{fig:silu}
\end{figure}

\end{document}
\end{minted}
\end{document}