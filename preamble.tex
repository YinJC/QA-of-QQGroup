\usepackage{zhlipsum} % 提供生成中文随机文本指令\zhlipsum[n],n:1--50
\usepackage{codeshow} % 提供显示代码及其结果的环境codeshow
\usepackage{graphicx} % 提供插图环境及指令
\graphicspath{{Pictures/}} % 设定图片位置
\usepackage{minted} % 提供代码抄录环境minted,行间单行抄录mint,行内抄录mintinline
\newcommand{\mtl}[1]{\mintinline{tex}|#1|}
\newcommand{\mt}[1]{\mint[breaklines]{tex}|#1|}
\usepackage{filecontents} % 提供导言区附加子文档的环境
\usepackage{amsmath} % 常用数学公式环境
\usepackage{mathtools} % 扩展了amsmath宏包
\usepackage{subfiles} % 支持主文档与子文档各自编译
\usepackage{tikz} % Required for drawing custom shapes
\usepackage{tkz-euclide} % 常用几何图库
\usepackage{colortbl} % 彩色表格
\usepackage{tabularx} % 绘制表格
\usepackage{xcolor} % Required for specifying colors by name
\definecolor{ocre}{RGB}{0,126,196} % Define the orange color used for highlighting throughout the book

%== RED

\definecolor{sthlmLightRed}{RGB}{254,222,237} % HEX #c40064
\newcommand{\csthlmLightRed}[1]{{\color{sthlmLightRed}{#1}}}
\definecolor{sthlmRed}{RGB}{196,0,100} % HEX #fedeed
\newcommand{\csthlmRed}[1]{{\color{sthlmRed}{#1}}}
%\definecolor{sthlmDarkRed}{RGB}{} % HEX #
%\newcommand{\csthlmDarkRed}[1]{{\color{sthlmDarkRed}{#1}}}

\usepackage{calc} % For simpler calculation - used for spacing the index letter headings correctly
\usepackage{makeidx} % Required to make an index
\makeindex % Tells LaTeX to create the files required for indexing

\newtheorem{qsT}{Q}[chapter] % 自定义qsT定理环境,显示为Q,计数器关联到章

\usepackage[framemethod=default]{mdframed} % Required for creating the theorem, definition, exercise and corollary boxes

% Essential Question box
\newmdenv[skipabove=7pt,
skipbelow=7pt,
rightline=false,
leftline=false,
topline=true,
bottomline=true,
backgroundcolor=sthlmRed!5,
linecolor=sthlmRed,
innerleftmargin=5pt,
innerrightmargin=5pt,
innertopmargin=5pt,
innerbottommargin=5pt,
leftmargin=0cm,
rightmargin=0cm,
linewidth=4pt]{essentialqBox}

\newenvironment{qst}%
{\begin{essentialqBox}\begin{qsT}}%
{\end{qsT}\end{essentialqBox}} % 自定义qst环境,使用如上的盒子与定理格式

\newtheorem{jd}{A:} % 自定义jd定理环境,显示为A:
\newcommand{\ans}{{\small\bf\sffamily\color{ocre}A:}\; } % 自定义解答指令ans,显示A:

%\newif\ifcompile\compilefalse
\newif\ifcompile\compiletrue

\usepackage{hyperref} % 提供交叉引用等的超链接功能
\hypersetup{%
	hidelinks,        % 隐藏链接,不显示链接的颜色及边框
%	backref=true,     % 反向链接,返回引用处所在节,默认只有正向引用
%	pagebackref=true, % 反向链接,返回引用处所在页,默认只有正向引用
%	hyperindex=true,  % 生成索引的页码链接,默认生成
	colorlinks=true,  % 链接改为彩色显示,默认为红色边框
	breaklinks=true,  % 允许链接折行,默认不允许
	urlcolor= ocre,   % 网址链接的颜色,默认为magenta
	linkcolor=blue,   % 页码和序号等链接的颜色,默认为red
%	bookmarks=true,   % 在PDF文档中创建书签目录,默认创建
	bookmarksopen=false, % 展开书签目录展开深度内所有章节标题,默认只展开章标题
	pdftitle={\LaTeX{}技术交流群问题汇总}, % 设置PDF文档属性中标题字段,默认为空
	pdfauthor={Group members} % 设置PDF文档属性中作者字段,默认为空
}
